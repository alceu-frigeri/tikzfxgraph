%%%==============================================================================
% !Mode:: "TeX:EN"
% Default Compile engines:
% !TEX program = pdflatex
% !PDFTeXify ext = --interaction=nonstopmode --enable-etex  --enable-write18
% !PDFLaTeX ext  = --interaction=nonstopmode --enable-etex  --enable-write18
% !BIB program = biber
%%%==============================================================================
%% Copyright 2025-present by Alceu Frigeri
%%
%% This work may be distributed and/or modified under the conditions of
%%
%% * The [LaTeX Project Public License](http://www.latex-project.org/lppl.txt),
%%   version 1.3c (or later), and/or
%% * The [GNU Affero General Public License](https://www.gnu.org/licenses/agpl-3.0.html),
%%   version 3 (or later)
%%
%% This work has the LPPL maintenance status *maintained*.
%%
%% The Current Maintainer of this work is Alceu Frigeri
%%
%% This is version {1.0} {2025/04/19}
%%
%% The list of files that compose this work can be found in the README.md file at
%% https://ctan.org/pkg/tikzfxgraph
%%
%%%==============================================================================
\documentclass[10pt]{article}
\RequirePackage[verbose,a4paper,marginparwidth=27.5mm,top=2.5cm,bottom=1.5cm,hmargin={40mm,20mm},marginparsep=2.5mm,columnsep=10mm,asymmetric]{geometry}
\usepackage{codedescribe}
\RequirePackage[inline]{enumitem}
\SetEnumitemKey{miditemsep}{parsep=0ex,itemsep=0.4ex}

\RequirePackage[backend=biber]{biblatex}
\addbibresource{tikzfxgraph.bib}

\usepackage{tikz}
\usetikzlibrary{matrix}
\usepackage{pgfplots}
\pgfplotsset{compat=1.18}
\usetikzlibrary{pgfplots.units}

\usepackage{tikzfxgraph}

\RequirePackage[hidelinks,hypertexnames=false]{hyperref}

\begin{document}
\tstitle{
  author={Alceu Frigeri\footnote{\tsverb{https://github.com/alceu-frigeri/tikzfxgraph}}},
  date={\tsdate},
  title={The tikzfxgraph Package\break simplified $f_i(x)$ graphics\break Version \PkgInfo{tikzfxgraph}{version}}
  }


\begin{typesetabstract}


This package offers a set of streamlined commands to draw algebraic functions, atop of \tsobj[pkg]{pgfplots,gnuplot}. Some auxiliary commands are also defined allowing to create \emph{sets of functions} and \emph{user defined styles}.
\end{typesetabstract}

\tableofcontents

\section{Introduction}

The \tsobj[pkg]{pgfplots}\cite{PGFPLOTS:1.18} package together with \tsobj[pkg]{gnuplot}\cite{gnuplot:5.4} is extremely flexible, allowing the construction of very intricate graphics, but said flexibility comes at a cost: the sheer number of parameters to be set.
This package is nothing more than a wrap around \tsobj[pkg]{pgfplots,gnuplot} offering a somewhat simplified interface.

A command (\tsobj{\fxgraphdraw}) and an environment (\tsobj[env]{fxgraph}) are provided for drawing one or more $f_i(x)$ curves, as well some auxiliary commands (\tsobj{\fxsetnew,\fxsetappend}) to define sets of functions and one for user defined styles (\tsobj{\fxsetnewstyle}).


\section{Requirements}

One needs to load at least two packages: \tsobj{tikz,pgfplots} (it is advisable to set the \tsobj[pkg]{pgfplots} compatibility to, at least, 1.18 (\tsverb{\pgfplotsset{compat=1.18}}). Like
\begin{codestore}[demo-pre]
\usepackage{tikz}
\usepackage{pgfplots}
\pgfplotsset{compat=1.18}

\usetikzlibrary{pgfplots.units} %this is optional
\end{codestore}
\tscode*{demo-pre}


Besides that the \tsobj[pkg]{gnuplot} must be installed/present.

\begin{tsremark}
to be able to call \tsobj[pkg]{gnuplot} you will need to use the \tsobj[key]{--enable-write18} (or \tsobj[key]{--shell-scape}) option in your \LaTeX\  run. Keep in mind that \tsobj[pkg]{gnuplot} will create a set of files that can be used from one run to another, and, unless you change the domain/functions, \tsobj[key]{--enable-write18} (or \tsobj[key]{--shell-scape}) can be kept disable otherwise.
\end{tsremark}

\begin{tsremark}[\color{red}Warning:]
As of now, there is a bug on version 6.0.*, under windows, of \tsobj[pkg]{gnuplot}. It will work, but a series of errors messages will be raised (invalid characters) that's due one of the returning files being written in UTF16 instead of UTF8.
So, for now, use at most version 5.4.*
\url{https://sourceforge.net/p/gnuplot/bugs/2747/}
\end{tsremark}


\section{Commands}

\subsection{Defining Functions' sets}\label{function-set}

\begin{codedescribe}{\fxsetnew}
\begin{codesyntax}%
\tsmacro{\fxsetnew}{new-fxset}
\end{codesyntax}
This defines/create a \tsobj[marg]{fxset} for later reference. A \tsobj[marg]{fxset} is just a ''repository'' of functions descriptions/specifications.
\end{codedescribe}

\begin{tsremark}
About dataset's names: It can be almost anything, the name can contain strings normally not allowed in a macro name, like spaces, dots, two-dots and so on, including backslashes, meaning that if someone typesets \tsobj{\XYZ} as a dataset, \tsobj{\XYZ} will be it's name: a backslash isn't an active character anymore and one can't use macros when defining a fxset's name.
\end{tsremark}


\begin{codedescribe}{\fxsetappend}
\begin{codesyntax}%
\tsmacro{\fxsetappend}{fxset,keyval-list}
\end{codesyntax}
Adds a function description to a given fxset.
\end{codedescribe}

Valid Keys when describing a function:
\begin{describelist*}{option}
\describe{fx}{The function itself. It can be any \tsobj[pkg]{gnuplot} valid expression, in terms of $x$.}
\describe{id}{(optional) An unique identifier. A set of auxiliary files are created by \tsobj[pkg]{gnuplot} using this as a name suffix. As per the \tsobj[pkg]{pgfplot} manual, they are used to determined if there is the need to re-run \tsobj[pkg]{gnuplot}. }
\describe{legend}{(optional) The name of the function, to appears as a Legend.}
\end{describelist*}
Besides those, any other \tsobj[pkg]{pgfplots} valid key can be used. (e.g. red , thick : the specific curve will be in red, using a thick line). For example:
\begin{codestore}[fxset-01]
\fxsetnew{set-A} %creating a new 'set'

\fxsetappend{set-A}{
  id=f-A0 ,
  fx=x^2-x+2 ,
  thick %this key comes from tikz/pgf
  }
\fxsetappend{set-A}{
  id=f-A1 ,
  fx=x^2+x+3 ,
  red %this key comes from tikz/pgf
  }
\end{codestore}
\tscode*[
  codeprefix={} ,
  texcs2={fxsetnew,fxsetappend,fxsetnewstyle,fxgraphdraw,fxgraph} ,
  emph={id,fx,legend,set} ,
  emph={linear,loglog,semilog,ticks,function} ,
  emph3={min,max,N,delta,units} ,
  ]{fxset-01}


\begin{tsremark}
An error is raised if \tsobj[marg]{fxset} isn't defined.
\end{tsremark}
\begin{tsremark}
Either set a legend for each and every function, or to none of them. Mixing styles (some with a legend, some without, will result in functions being wrongfully labelled).
\end{tsremark}


\subsection{User Defined Styles}\label{style-def}

\begin{codedescribe}{\fxsetnewstyle}
\begin{codesyntax}%
\tsmacro{\fxsetnewstyle}{style-name,keyval-list}
\end{codesyntax}
Defines a new \tsobj[pkg]{pgfplot} style key. Which can be later used when drawing $f_i(x)$ function graphs (can be used, for instance, to assure all graphs follow the same style).
\end{codedescribe}


Valid Keys when describing a function:
\begin{describelist*}{option}
\describe{linear}{Both $x$ and $y$ axis are linear.}
\describe{loglog}{Both $x$ and $y$ axis are logarithmic.}
\describe{semilog x}{The $x$ axis is logarithmic, the $y$ is linear.}
\describe{semilog y}{The $y$ axis is logarithmic, the $x$ is linear.}
\describe{x ticks}{Describes the minor ticks to be drawn in the $x$ axis. See below.}
\describe{y ticks}{Describes the minor ticks to be drawn in the $y$ axis. See below.}
\end{describelist*}
Besides those, any other \tsobj[pkg]{pgfplots} valid key can be used. (e.g. red , thick : the specific style will set the lines to be red and thick).


Note that the keys \tsobj[key]{x ticks, y ticks} are themselves defined by a set of keyval values, as follow:
\begin{describelist*}{option}
\describe{min}{The minimal value (starting value) of the corresponding axis.}
\describe{max}{It`s maximum value.}
\describe{delta}{(optional) the $delta$ to be used between \emph{ticks}. Note that, it depends on the kind of the axis. In case of a linear axis, this is just the delta between ticks. If logarithmic, it is the geometric distantce between ticks.}
\describe{N}{(optional) Sets de number of ticks to be calculated. If case of a linear axis it will set the linear distance to $({max - min})/{N}$. In case of a logarithmic axis it will set the geometric distance to $(\ln(max) - \ln(min))/{N}$. \tsobj[key]{N} has precedence over \tsobj[key]{delta}.}
\describe{units}{(optional) The units of the corresponding axis.}
\end{describelist*}

The following example will define a style ''my style A'', to be used in a ''semilog x'' graph. The $x$ domain will go from $0.001$ up to $100$ with ticks at $0.001 ,\, 0.01 ,\, 0.1 ,\, 1.0 ,\, 10 \,\textrm{and}\,  100$. Conversely, the $y$ domain will go from $-\pi$ up to $+\pi$, with linearly spaced ticks. The ticks will be inside the graph.
\begin{codestore}[fxset-02]
\fxsetnewstyle{my style A}{
  semilog x ,
  x ticks = {
    min = 0.001 ,
    max = 100 ,
    N = 6 ,
    units = rad/s ,
    } ,
  y ticks = {
    min = -3.14159265 ,
    max =  3.14159265 ,
    N = 8 ,
    units = rad ,
    } ,
      % the following key comes from pgfplot
  tick align=inside , %this package's default is outside.
  }
\end{codestore}
\tscode*[
  codeprefix={} ,
  codeprefix={} ,
  texcs2={fxsetnew,fxsetappend,fxsetnewstyle,fxgraphdraw,fxgraph} ,
  emph={id,fx,legend,set} ,
  emph={linear,loglog,semilog,ticks,function} ,
  emph3={min,max,N,delta,units} ,
  ]{fxset-02}


\begin{tsremark}
The  \tsobj[key]{linear,loglog, semilog x, semilog y} keys are only used when setting the ticks (linear or logarithmic). If none is given, it is assumed that both $x$ and $y$ axis are linear.
\end{tsremark}
\begin{tsremark}
If either \tsobj[key,sep=or]{min,max} are missing, no tick list will be generated.
\end{tsremark}

\begin{tsremark}
In case of a logarithmic axis, both \tsobj[key]{min,max} must greater than zero, otherwise an error will be raised.
\end{tsremark}



\subsection{Drawing a $f_i(x)$ Graph}
There is a single drawing command \tsobj{\fxgraphdraw} and a companion environment \tsobj[env]{fxgraph}, both share the same interface. The graph will be constructed as follow: \begin{enumerate*}
                                                           \item An outer \tsobj[env]{tikzpicture} environment
                                                           \item An inner \tsobj[env]{axis} environment
                                                           \item The function's graphs themself
                                                           \item (in case of the \tsobj[env]{fxgraph} environment)  futher \tsobj[pkg]{pgfplot} commands.
                                                         \end{enumerate*}
The \tsobj[env]{axis} environment will first be ''styled'' (as per \tsobj[key,sep=or]{linear,loglog,semilog x, semilog y}, see \ref{style-default}) then the \tsobj[key]{ticks}, if defined, will be applied, lastly any further \tsobj[pkg]{pgfplot} key used when calling those commands.
\begin{tsremark}
Given the above construct, generic \tsobj[pkg]{pgfplot} keys used will always have a precedence over the default styles, regardless of they order of appearance.
\end{tsremark}

\begin{codedescribe}{\fxgraphdraw}
\begin{codesyntax}%
\tsmacro{\fxgraphdraw}{keyval-list}
\end{codesyntax}
Creates a graph, and draw one or more functions/sets of functions as describer by \tsobj[marg]{keval-list} (see below).
\end{codedescribe}

\begin{codedescribe}[env]{fxgraph}
\begin{codesyntax}%
\tsmacro{\begin{fxgraph}}{keyval-list}
\tsobj[oarg]{further commands}
\tsmacro{\end{fxgraph}}{}
\end{codesyntax}
Same as \tsobj{\fxgraphdraw}, allowing to add further \tsobj[pkg]{pgfplot,tikz} commands.
\end{codedescribe}

Valid Keys when describing a graph:
\begin{describelist*}{option}
\describe{linear}{Both $x$ and $y$ axis are linear.}
\describe{loglog}{Both $x$ and $y$ axis are logarithmic.}
\describe{semilog x}{The $x$ axis is logarithmic, the $y$ is linear.}
\describe{semilog y}{The $y$ axis is logarithmic, the $x$ is linear.}

\describe{x ticks}{Describes the minor ticks to be drawn in the $x$ axis. See below.}
\describe{y ticks}{Describes the minor ticks to be drawn in the $y$ axis. See below.}

\describe{sans tikzpicture}{Suppress the outer \tsobj[env]{tikpicture} environment.}
\describe{without tikzpicture}{Suppress the outer \tsobj[env]{tikpicture} environment.}

\describe{function}{Adds a function specification, see \ref{function-set}.}
\describe{fx set}{A command separated list of \tsobj[key]{fxsets}.}
\end{describelist*}
Besides those, any other \tsobj[pkg]{pgfplots} valid key can be used. (e.g. red , thick : the specific style will set the lines to be red and thick).

The \tsobj[key]{x ticks,y ticks} are set the same way as in \ref{style-def} (\tsobj[key]{x tick}=\tsobj[marg]{keyval-list}). 

The \tsobj[key]{function} key defines (as in \ref{function-set}) a function to be draw, it can be used multiple times. Note that those functions will be drawn before any \tsobj[key]{fx set}.
%\fxgraphdraw{semilog x={red},x ticks={min=0.1,max=1000,N=8},y ticks={min=1,max=100,N=10},function={{fX,red,x^2+3}}}

\tsobj[key]{fx set} is a comma separated list of \tsobj[marg]{fxset} (as defined in \ref{function-set}). All functions $f_i(x)$ associated with each \tsobj[marg]{fxset} will be drawn.

Normally, the \tsobj{\fxgraphdraw} command (viz-à-viz \tsobj[env]{fxgraph} environment) will insert an axis environment inside a \tsobj[env]{tizpicture} environment. The \tsobj[key]{sans tikzpicture,without tikzpicture} keys will suppress that external \tsobj[env]{tizpicture} environment.

\section{Style's Default}\label{style-default}

For each kind of graph (linear, loglog, semilog) there is a corresponding predefined style:

\begin{describelist*}{key}
\describe{linear axis}{This is the ``base'' style. Both axis will be gridded (lines at the corresponding ticks), with an axis line at the bottom (for $x$) and the left (for $y$). Ticks marks will be outside the graph. The legend (if present) will be at the top right of the graph. The $f_i(x)$ curves will be styled according to two lists: \tsobj[key]{fxgraph color list, fxgraph line list} (see below). The graph width is set to 80\% of \tsobj{\textwidth} and it's height to 35\% of \tsobj{\textwidth}.}
\describe{loglog axis}{It applies the current \tsobj[key]{linear axis} style and the log basis is set to 10.}
\describe{semilog x axis}{It applies the current \tsobj[key]{linear axis} style and the log basis is set to 10.}
\describe{semilog y axis}{It applies the current \tsobj[key]{linear axis} style and the log basis is set to 10.}
\end{describelist*}

Those styles can be modified with \tsobj[code,sep=or]{\pgfplotsset,\pgfkeys} (using \tsobj[key]{.style append} sub-key, for instance. See \cite{TIKZPGF:3.1} and \cite{PGFPLOTS:1.18}). If using the \tsobj{\pgfkeys} remember to switch first to the \tsobj[key]{pgfplots} `family'.

When styling a set of functions, two lists are used (\tsobj[key]{cycle multiindex* list} from \tsobj[pkg]{pgfplots}):

\begin{describelist*}{key}
\describe{fxgraph color list}{Function's color will cycle through \tsobj[key]{red!80!black,
    green!80!black,
    blue!80!black,
    black,
    brown!70!black,
    teal!80!black,
    orange!80!black,
    violet!80!black,
    cyan!80!black,
    magenta!80!black,
    yellow!75!black,
    black!60!white}. }
\describe{fxgraph line list}{Function's line style will cycle through \tsobj[key]{    solid,
    solid,
    solid,
    dashed,
    dashed,
    dashed,
    dashdotdotted,
    dashdotdotted,
    dashdotdotted}.}
\end{describelist*}

Both can be redefined with \tsobj{\pgfplotscreateplotcyclelist} from \tsobj[pkg]{pgfplots}.




\section{Examples}\label{Examples}

\subsection{Drawing a Bode Diagram}
%
\begin{codestore}[fxset-03]
\fxsetnewstyle{db style}{
  semilog x ,
  y ticks = { min = -20 , max =  80 , N = 5 , units = db } ,
  }

\fxsetnewstyle{phi style}{
  semilog x ,
  y ticks = { min = -3.14159265 , max =  3.14159265 , N = 8 , units = rad } ,
  }

\fxsetnewstyle{freq range A}{
  semilog x ,
  x ticks = { min = 0.01 , max =  100000 , N = 7 , units = rad/s } ,
  }
%
%%% This is optional, just defining an auxiliary macro with f(x) core expression
%%% Note that this is a valid gnuplot's expression (not LaTeX/TeX/...)
\def\Hs{(x*{0,1}+1)*(x*{0,1}+40000)/((-x^2+0.01*2*400*x*{0,1}+400^2))}

\fxgraphdraw{
  semilog x ,
  db style ,
  freq range A ,
  function={fx=20*log10(abs(\Hs))}
  }

\fxgraphdraw{
  semilog x ,
  phi style ,
  freq range A ,
  function={fx={atan2( imag(\Hs) , real(\Hs) )}}
  }

\end{codestore}
\tsdemo*[
  codeprefix={} ,
  resultprefix={} ,
  texcs2={fxsetnew,fxsetappend,fxsetnewstyle,fxgraphdraw,fxgraph} ,
  emph={id,fx,legend,set} ,
  emph={linear,loglog,semilog,ticks,function} ,
  emph3={min,max,N,delta,units} ,
  ]{fxset-03}


\subsection{A Few Curves at Once}
In the example below note that the \tsobj[key]{fx set} functions are drawn after the 'Dx'. And that defines the legend order. 
\begin{codestore}[fxset-04]
\fxsetnew{set-B}
\fxsetappend{set-B}{id=B0,fx=2*cos(x)+1,legend=B0}
\fxsetappend{set-B}{id=B1,fx=2*sin(2*x)-1,legend=B1,thick} 
            %the thick (line) keyword comes from tikz
\fxsetnew{set-C}
\fxsetappend{set-C}{id=C0,fx=1.5*cos(x+2)+1,legend=C0}
\fxsetappend{set-C}{id=C1,fx=2*sin(x-2)-1,legend=C1}

\fxgraphdraw{
    linear ,
    y ticks = {min = -3 , max = 3 , N = 6} ,
    x ticks = {min = 0 , max = 3*3.14159265 , N = 6 , units = rad} ,
    fx set = {set-B , set-C} ,
    function = {id=Dx,fx=x-2,legend=Dx} ,
    xlabel = some radians ,  % from pgfplots
    ylabel = This y ,        % from pgfplots
    title = Graph A          % from pgfplots
}

\fxgraphdraw{
    linear ,
    y ticks = {min = -3 , max = 3 , N = 6} ,
    x ticks = {min = 0 , max = 3*3.14159265 , N = 6  , units = rad} ,
    fx set = {set-C} ,
    xlabel = some more , % from pgfplots
    ylabel = This y ,    % from pgfplots
    title = Graph B      % from pgfplots
}

\end{codestore}
\tsdemo*[
  codeprefix={} ,
  resultprefix={} ,
  texcs2={fxsetnew,fxsetappend,fxsetnewstyle,fxgraphdraw,fxgraph} ,
  emph={id,fx,legend,set} ,
  emph={linear,loglog,semilog,ticks,function} ,
  emph3={min,max,N,delta,units} ,
  ]{fxset-04}

\newpage
\subsection{Further Customization}
There are many ways, for instance, to have side by side graphs. One could use, for example, a \tsobj[env]{tabular} environment. In the following the \tsobj[pkg]{tikz} library \tsobj[pkg]{matrix} will be used, in which case the option \tsobj[key]{sans tikzpicture} is needed. Furthermore, it is needed to customize the width and height of each graph. Note: \tsobj[pkg]{tikz} matrix cannot be nested, and since \tsobj[pkg]{pgfplots} legend are created as a \tsobj[pkg]{tikz} matrix, one can't have a legend in this case.

\begin{codestore}[fxset-05]
\fxsetnew{set-D}
\fxsetappend{set-D}{id=B0,fx=2*cos(x)+1}
\fxsetappend{set-D}{id=B1,fx=2*sin(2*x)-1,thick}
            %the thick (line) keyword comes from tikz
\fxsetnew{set-E}
\fxsetappend{set-E}{id=C0,fx=1.5*cos(x+2)+1}
\fxsetappend{set-E}{id=C1,fx=2*sin(x-2)-1}

\begin{tikzpicture}
\matrix{
\fxgraphdraw{
    linear ,
    y ticks = {min = -3 , max = 3 , N = 6} ,
    x ticks = {min = 0 , max = 3*3.14159265 , N = 6} ,
    fx set = {set-D} ,
    function = {id=Fx,fx=x-2} ,
    sans tikzpicture ,
    width=0.47\textwidth,   % from pgfplots
    height=0.30\textwidth , % from pgfplots
} &
\fxgraphdraw{
    linear ,
    y ticks = {min = -3 , max = 3 , N = 6} ,
    x ticks = {min = 0 , max = 3*3.14159265 , N = 6} ,
    fx set = {set-E} ,
    function = {id=Gx,fx=x-2} ,
    sans tikzpicture ,
    width=0.47\textwidth,    % from pgfplots
    height=0.30\textwidth ,  % from pgfplots
} \\
};
\end{tikzpicture}
 
\end{codestore}
\tscode*[
  codeprefix={} ,
  resultprefix={} ,
  texcs2={fxsetnew,fxsetappend,fxsetnewstyle,fxgraphdraw,fxgraph} ,
  emph={id,fx,legend,set} ,
  emph={linear,loglog,semilog,ticks,function} ,
  emph3={min,max,N,delta,units} ,
  ]{fxset-05}

\tsresult*[
  codeprefix={} ,
  resultprefix={} ,
  texcs2={fxsetnew,fxsetappend,fxsetnewstyle,fxgraphdraw,fxgraph} ,
  emph={id,fx,legend,set} ,
  emph={linear,loglog,semilog,ticks,function} ,
  emph3={min,max,N,delta,units} ,
  ]{fxset-05}



\printbibliography

\end{document} 