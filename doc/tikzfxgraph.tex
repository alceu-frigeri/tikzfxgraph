% !TEX program = pdflatex
% !TEX ext =  --interaction=nonstopmode --enable-etex --enable-write18
% !BIB program = none
%%%==============================================================================
%% Copyright 2025-present by Alceu Frigeri
%%
%% This work may be distributed and/or modified under the conditions of
%%
%% * The [LaTeX Project Public License](http://www.latex-project.org/lppl.txt),
%%   version 1.3c (or later), and/or
%% * The [GNU Affero General Public License](https://www.gnu.org/licenses/agpl-3.0.html),
%%   version 3 (or later)
%%
%% This work has the LPPL maintenance status *maintained*.
%%
%% The Current Maintainer of this work is Alceu Frigeri
%%
%% This is version {*1.0*} {20**/**/**}
%%
%% The list of files that compose this work can be found in the README.md file at
%% https://ctan.org/pkg/tikzfxgraph
%%
%%%==============================================================================
\documentclass[10pt]{article}
\RequirePackage[verbose,a4paper,marginparwidth=27.5mm,top=2.5cm,bottom=1.5cm,hmargin={40mm,20mm},marginparsep=2.5mm,columnsep=10mm,asymmetric]{geometry}
\usepackage{codedescribe}
\RequirePackage[inline]{enumitem}
\SetEnumitemKey{miditemsep}{parsep=0ex,itemsep=0.4ex}

%%\usepackage[american,siunitx,cuteinductors,smartlabels,arrowmos,EFvoltages,betterproportions]{circuitikz}
%%\usetikzlibrary{math}
%%\usepackage{tikzfxgraph}

\RequirePackage[hidelinks,hypertexnames=false]{hyperref}

\begin{document}
\tstitle{
  author={Alceu Frigeri\footnote{\tsverb{https://github.com/alceu-frigeri/tikzfxgraph}}},
  date={\tsdate},
  title={The tikzfxgraph Package\break ***Something*** \break Version \PkgInfo{tikzfxgraph}{version}}
  }
  

\begin{typesetabstract}
 
This package offers 
\end{typesetabstract}

%\tableofcontents

\section{Introduction}

\begin{codedescribe}{\somecmd,\somecmd}
\begin{codesyntax}%
\tsmacro{\somecmd}{}
\tsmacro{\somecmd}{}
\end{codesyntax}
These will affect how
\end{codedescribe}
\begin{tsremark}[\color{red}Warning:]
  The default ...
\end{tsremark}


\begin{codestore}[demoA]
some code
\end{codestore}

%\tscode
%\tsdemo*[emph={draw,none},emph2={path},emph3={name},basicstyle={\scriptsize\ttfamily},numbers=left]{demoA}
%\tsresult



\end{document} 